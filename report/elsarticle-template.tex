%%
%% Copyright 2007, 2008, 2009 Elsevier Ltd
%%
%% This file is part of the 'Elsarticle Bundle'.
%% ---------------------------------------------
%%
%% It may be distributed under the conditions of the LaTeX Project Public
%% License, either version 1.2 of this license or (at your option) any
%% later version.  The latest version of this license is in
%%    http://www.latex-project.org/lppl.txt
%% and version 1.2 or later is part of all distributions of LaTeX
%% version 1999/12/01 or later.
%%
%% The list of all files belonging to the 'Elsarticle Bundle' is
%% given in the file `manifest.txt'.
%%

%% Template article for Elsevier's document class `elsarticle'
%% with numbered style bibliographic references
%% SP 2008/03/01
%%
%%
%%
%% $Id: elsarticle-template-num.tex 4 2009-10-24 08:22:58Z rishi $
%%
%%
\documentclass[preprint,12pt,3p]{elsarticle}

%\documentclass[final,3p,times]{elsarticle}
%% Use the option review to obtain double line spacing
%% \documentclass[preprint,review,12pt]{elsarticle}

%% Use the options 1p,twocolumn; 3p; 3p,twocolumn; 5p; or 5p,twocolumn
%% for a journal layout:
%% \documentclass[final,1p,times]{elsarticle}
%% \documentclass[final,1p,times,twocolumn]{elsarticle}
%% \documentclass[final,3p,times]{elsarticle}
%% \documentclass[final,3p,times,twocolumn]{elsarticle}
%% \documentclass[final,5p,times]{elsarticle}
%% \documentclass[final,5p,times,twocolumn]{elsarticle}

%% if you use PostScript figures in your article
%% use the graphics package for simple commands
%% \usepackage{graphics}
%% or use the graphicx package for more complicated commands
%% \usepackage{graphicx}
%% or use the epsfig package if you prefer to use the old commands
%% \usepackage{epsfig}

%% The amssymb package provides various useful mathematical symbols
\usepackage{amssymb}
\usepackage{amsmath}
\usepackage{graphicx}
\usepackage{enumitem}
\usepackage{mathtools}
\usepackage[utf8]{inputenc}
\usepackage{color}
\usepackage[procnames]{listings}
\usepackage[]{algorithm2e}


\graphicspath{ {img/} }
%% The amsthm package provides extended theorem environments
%% \usepackage{amsthm}

%% The lineno packages adds line numbers. Start line numbering with
%% \begin{linenumbers}, end it with \end{linenumbers}. Or switch it on
%% for the whole article with \linenumbers after \end{frontmatter}.
%% \usepackage{lineno}

%% natbib.sty is loaded by default. However, natbib options can be
%% provided with \biboptions{...} command. Following options are
%% valid:

%%   round  -  round parentheses are used (default)
%%   square -  square brackets are used   [option]
%%   curly  -  curly braces are used      {option}
%%   angle  -  angle brackets are used    <option>
%%   semicolon  -  multiple citations separated by semi-colon
%%   colon  - same as semicolon, an earlier confusion
%%   comma  -  separated by comma
%%   numbers-  selects numerical citations
%%   super  -  numerical citations as superscripts
%%   sort   -  sorts multiple citations according to order in ref. list
%%   sort&compress   -  like sort, but also compresses numerical citations
%%   compress - compresses without sorting
%%
%% \biboptions{comma,round}

% \biboptions{}


\begin{document}

\begin{frontmatter}

\title{icc: iMSI catcher catcher}

\author{Jan Kuipers}
\address{j.h.kuipers@student.utwente.nl}
\address{University of Twente}
\author{David Stritzl}
\address{david.stritzl@gmail.com}
\address{University of Twente}
\author{Santiago Aragón}
\address{s.e.aragonramirez@student.utwente.nl}
\address{University of Twente}
\author{Iwan Timmer}
\address{i.r.timmer@student.utwente.nl}
\address{University of Twente}
\begin{abstract}
This program tries to find nearby IMSI catchers using a RTL\_SDR device. TODO: Diagram
\end{abstract}


\end{frontmatter}

%%
%% Start line numbering here if you want
%%
% \linenumbers2

%% main text



\section{Motivation and Outline}
What is an IMSI Catcher
Why to spoit it
Possible approaches and difficculties
Our approach
Outline
\section{Design}

\subsection{Detection methods}

Detections methods (DM) are defined as python scripts in detectors/some\_dector.py. Every method should extend the class Detector specified in detectors/Detector.py and define its own callback function, e.g.:
\lstset{language=Python}
\begin{lstlisting}
    def handle_packet(self, data):
        p = GSMTap(data)

        if p.payload.name is 'LAPDm' and
           p.payload.payload.name is 'GSMAIFDTAP' and
           p.payload.payload.payload.name is 'CipherModeCommand':
                cipher = p.payload.payload.payload.cipher_mode >> 1

                if cipher == 0:
                    self.update_s_rank(Detector.SUSPICIOUS)
                    self.comment = 'A5/1 detected'
                ...

\end{lstlisting}

This function will be applied packet wise and should rank the anylyzed BTS and at the end modify the \texttt{s\_rank} and \texttt{comment} variables calling \texttt{self.update\_s\_rank(RANK)}(resp. \texttt{self.comment='A descriptive comment'}).

We define rank the suspiciones of a BTS as
\begin{lstlisting}

    SUSPICIOUS = 2
    UNKNOWN = 1
    NOT_SUSPICIOUS = 0
\end{lstlisting}

At the end of the detection the detectors return a \texttt{TowerRank} object.




\section{Implementation details}
\section{Limitations and future work}



%% References
%%
%% Following citation commands can be used in the body text:
%% Usage of \cite is as follows:
%%   \cite{key}         ==>>  [#]
%%   \cite[chap. 2]{key} ==>> [#, chap. 2]
%%
%% References with bibTeX database:

\bibliographystyle{elsarticle-num}
% % \bibliographystyle{elsarticle-harv}
% % \bibliographystyle{elsarticle-num-names}
% % \bibliographystyle{model1a-num-names}
% % \bibliographystyle{model1b-num-names}
% % \bibliographystyle{model1c-num-names}
% % \bibliographystyle{model1-num-names}
% % \bibliographystyle{model2-names}
% % \bibliographystyle{model3a-num-names}
% % \bibliographystyle{model3-num-names}
% % \bibliographystyle{model4-names}
% % \bibliographystyle{model5-names}
% % \bibliographystyle{model6-num-names}
\bibliography{sample}


\end{document}

%%
%% End of file `elsarticle-template-num.tex'.
